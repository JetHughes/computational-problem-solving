\documentclass{homework}
\usepackage{listings}
\usepackage{color}
\definecolor{dkgreen}{rgb}{0,0.6,0}
\definecolor{gray}{rgb}{0.5,0.5,0.5}
\definecolor{mauve}{rgb}{0.58,0,0.82}
\usepackage{graphicx}
\graphicspath{ {./images/} }

\lstset{frame=tb,
  language=Python,
  aboveskip=3mm,
  belowskip=3mm,
  showstringspaces=false,
  columns=flexible,
  basicstyle={\small\ttfamily},
  numbers=none,
  numberstyle=\tiny\color{gray},
  keywordstyle=\color{blue},
  commentstyle=\color{dkgreen},
  stringstyle=\color{mauve},
  breaklines=true,
  breakatwhitespace=true,
  tabsize=3
}

% Put in your Name !
\newcommand{\hwname}{Callum Teape, Jet Hughes, Jake Norton, Cody Airey}



% Header
\newcommand{\hwtype}{Desert Crossing}
\newcommand{\hwclass}{}
\newcommand{\hwnum}{}

%%% Define shortcuts here
\newcommand{\B}{\mathcal B}
% instead of typing \mathcal B to display a curly B I can now just type \B
\newcommand{\ip}[2]{\left\langle #1, #2 \right\rangle}
% this defines the new command \ip{x}{y} to display <x,y>

\newcommand*{\N}{\mathbb N}
\newcommand*{\Z}{\mathbb Z}
\newcommand*{\Q}{\mathbb Q}
\newcommand*{\R}{\mathbb R}
\newcommand*{\C}{\mathbb C}
\newcommand{\F}{\mathbb F}
\newcommand{\norm}[1]{\|  #1 \|}
%%%%
\begin{document}

\maketitle


\question Using the vehicle without refuelling, how far into the desert can you safely go?

Without refuelling, the vehicle has a capacity of 60L + 4x20L, which is 140L. The vehicle can travel 12km per 1L of fuel. This means without refuelling the vehicle can travel 12km x 140L = 1680km. 

\question Describe a procedure whereby you could cross the desert in the vehicle.

To cross the desert, you could make several trips out into the desert, to drop fuel at "checkpoints" along the route. With enough fuel ready at points spaced close enough, you could make a trip across the desert by refuelling at the checkpoints along the way.

\question Describe a procedure whereby you could cross the desert and return in the vehicle.

To cross the desert and return, you could virtually the same procedure as in question 2. In order to facilitate a return trip, you would need to drop enough extra fuel at the checkpoints to be able to make a return trip while refuelling along the way.

\question Describe a procedure whereby you could cross the desert in the vehicle using the minimum amount of fuel.
\newline\newline
Firstly, we note that the further into the desert you drop fuel, the amount of fuel you use getting there and back increases. So there is clearly some sort of cost associated with dropping fuel further out into the desert. With this in mind, we seek to maximise the distance between the fuel depot locations and the end of the desert. This leads us to a few critical assumptions:

\begin{enumerate}
    \item We assume that the optimal solution will have the final fuel depot one full tank away from the end, as this is the maximum possible distance away from the end of the desert.
    \item We assume that by the time we're done, there should be no fuel left in the desert. If there is then that means we've wasted fuel, and so our solution is not minimal.
    \item We assume that every time we leave a depot driving out into the desert, we have full tank. This seems logical, as if we are going to be dropping fuel at the next depot, then we have to bear the cost of driving there, and we want to maximise the amount of fuel we drop per unit of cost to get the most out of every mile we drive. It doesn't make any sense to have only half a tank and then only be able to drop 20 litres, when we could have had a full tank and dropped 80 litres for the same driving cost.
    \item We drop fuel at only one depot per trip from base. This assumption is more or less a consequence of 3), as it doesn't make sense to drop fuel at a location and then fuel up to a full tank to drive to another. This is because there is more driving cost associated with this move then simply leaving more fuel there in a previous trip (where we didn't drive as far from base).
    \item As a consequence of 4), we also assume that we drop fuel at each depot only once. At a distance of this range, this assumption is ok. If the desert was much longer it wouldn't be possible to do this. This assumption seems logical, because we only drop fuel at one depot per trip, so two drop fuel at a depot twice means taking two whole separate trips to do it, which by 3) involves two refuels at base.
    
\end{enumerate}
Using these assumptions, we try to find a solution.
\newline
Let $(a_n) = (a_1,a_2,...a_n)$ be the locations of the fuel depots in litres from the start of the desert.
\newline\newline
By 1), we fix $a_n$ at $\frac{d-mc}{m}$
\newline\newline
Now we are left with two questions: How many fuel depot's should we use between the start of the desert and the final drop? And how should these fuel depot's be distributed?
\newline\newline
4) essentially gives us that we want the minimum possible fuel drops such that 5) holds.
\newline\newline
3) give us that we need to have $d(a_n, a_{n-1})$ at the last depot. If we only drop fuel here once, and we are trying to maximise the distance between $a_{n-1}$ and the end of the desert, then this give us that we have $3*d(a_n, a_{n-1}) = c$. so $d(a_n, a_{n-1}) = \frac{c}{3}$. 
\newline\newline
For $d = 2413, c = 140, m = 12$, this gives us $(a_n) = (0, 14.417, 61.083)$. 
\newline\newline
So we cross the desert as follows:
\newline
\begin{enumerate}
    \item Start with a full tank (+140L) go 14.137 litres outs, drop $3*14.137$ litres and return. Arrive back at base with 69.315 litres.
    \item Refuel (+70.685) go 14.137 litres, fuel up 14.137 litres, drive a further 46.666 litres out to 61.083 litres, drop 46.666 litres, drive back to 14.137 litres, we're empty at this point, so pick up 14.137 litres and drive back to base.
    \item Refuel (+140), drive 14.137 litres out, pick up all the remaining 14.137 litres so we have a full tank again. Drive to 61.083 litres, pick up the 46.666 litres there so we have a full tank again. Drive all the way to the end of the desert.
\end{enumerate}
The total expenditure for this trip is 140 + 70.685 + 140 = 350.685 litres
\newpage

\question Describe a procedure whereby you could cross the desert and return in the vehicle using the minimum amount of fuel.
\newline\newline
Let the total length of the desert be $d$, let the mileage of the vehicle be $m$ in km per litre, let the total fuel capacity of the vehicle be $c$.
\newline\newline
We start by seeking some sort of standard procedure to cross the desert. With this we can find some sort of general formula for expenditure, and therefore can evaluate solutions programmatically rather than by hand. So we start with a few assumptions, which we came by logic and trial and error:

\begin{enumerate}
    \item As before, we assume that the best solution has it's final fuel depot the maximum possible distance from the end of the desert, which in this case is half a tank. Which gives us that we need to have a full tank when we depart this depot for the end of the desert.
    \item 2 from Q4)
    \item 3 from Q4)
    \item 4 from Q4)
    \item On the final return journey, we assume that the optimum solution will have us run out of fuel exactly as we come to each fuel depot, which in turn gives us that to make it back, we need to leave at each fuel depot the exact amount of fuel that it takes to drive to the next. This seems logical, as it means we arrive back at base empty, implying we used the minimal fuel.

\end{enumerate}
With these assumptions, we can work backwards from the finish line to construct a general procedure, and therefore a general expenditure. To simplify the explanation, let $(a_n) = (a_1, a_2, ... a_n)$ be the locations of the fuel depots in litres from the start of the desert, let $c, m, d$ be as defined before.
\newline\newline
By 1), we need a full tank when we leave the final depot for the end of the desert. By 4), we know that we have a full tank every time we leave the depot before the final depot so to have a full tank when we leave for the end, we will need to have $d(a_n, a_{n-1})$ litres there to refuel what we lost by driving there. By 2), we know that we will need to leave another $d(a_n, a_{n-1})$ litres at the final depot for the return journey. So we need $2*d(a_n, a_{n-1})$ at the final depot.
\newline\newline
By 4), we have a full tank every time we leave depot $a_{n-1}$, so we can drop $min(80, c - 2*d(a_n, a_{n-1})$, litres at the final depot with every trip we make from the depot before. Therefore we need to make $\frac{2*d(a_n, a_{n-1})}{min(80, c - 2*d(a_n, a_{n-1})}$ trips from depot $a_{n-1}$ in order to drop enough fuel at the last depot.
\newline\newline
By 4) we have a full tank every time we leave depot $a_{n-2}$, so to ensure that 4) holds for $a_{n-1}$, we need to leave enough there to refuel the amount lost by driving from the depot before however many times we will pass it. We know that we have to make $\frac{2*d(a_n, a_{n-1})}{min(80, c - 2*d(a_n, a_{n-1})}$ trips beyond depot $a_{n-1}$ to ensure there is enough fuel at the last depot for the final journey, and we also know that will pass depot $a_{n-1}$ twice more (there and back) for that final journey. So the amount of fuel needed at depot $a_{n-1}$ is $(2*\frac{2*d(a_n, a_{n-1})}{min(80, c - 2*d(a_n, a_{n-1})} + 2)*d(a_{n-1},a_{n-2})$.
\newline\newline
Now, by the same logic, we will need to make 
\newline\begin{center}
\large $\frac{(2*\frac{2*d(a_n, a_{n-1})}{min(80, c - 2*d(a_n, a_{n-1})} + 2)*d(a_{n-1},a_{n-2})}{min(80, c - 2*d(a_{n-1},a_{n-2}))}$ 
\newline\end{center}
trips from depot $a_{n-2}$ in order to drop of enough fuel, but we also will pass it $2*\frac{2*d(a_n, a_{n-1})}{min(80, c - 2*d(a_n, a_{n-1})}$ more when we make journeys from depot $a_{n-1}$ to drop fuel at the final depot, and we also need to leave $2*d(a_{n-2},a_{n-3})$ for the final journey. So we require 
\newline\begin{center}
\large $(2*\frac{(2*\frac{2*(d(a_n, a_{n-1})}{min(80, c - 2*d(a_n, a_{n-1})} + 2)*d(a_{n-1},a_{n-2})}{min(80, c - 2*d(a_{n-1},a_{n-2}))} + 2*\frac{2*d(a_n, a_{n-1})}{min(80, c - 2*d(a_n, a_{n-1})} + 2)*d(a_{n-2},a_{n-3})$
\newline\end{center}
litres of fuel at depot $a_{n-2}$. 
\newline\newline
We can express the logic above more concisely by iteratively defining a sequences $(r_{n\in\N})$, for the required fuel at each depot, and $(t_{n\in\N})$ for the number of trips from each depot. Then with initial conditions $r_0 = 2*d(a_n, a_{n-1})$ and $t_0 = \frac{2*d(a_n, a_{n-1})}{min(80, c - 2*d(a_n, a_{n-1})}$. 
\newline\begin{center}
\large$r_i = d(a_{n-(i+1)}, a_{n-i})(2(\sum_{k=0}^{i-1}t_k)+2)$\newline\newline
\large$t_i = \frac{r_i}{min(80,c-2*d(a_{n-i},a_{n-(i+1)})}$ rounded strictly up to nearest whole number\newline
\end{center}
Note that the length of the $(r)$ and $(t)$ will end up being one less than the length of $(a)$. We obtain a general formula for expenditure where n is the length of the array $(a)$:
\begin{center}
\large $E = c + \sum_{i=0}^{n-1}[t_i*d(a_{n-i},a_{n-(i+1)}) + r_i]$
\end{center}
Finally, we can codify this in the following method:

\begin{lstlisting}
def calcExpenditure5(a, c):
    """
    calculating the expenditure given an array of fuel drop locations for round trip.

    :params a: integer array
        the array of fuel drop positions in litres from the start of the desert
    """
    numTrips = 0

    tripcost = []   #array which holds the cost of each leg of the journey

    n = len(a)  

    for k in range(1,n):   
        j = n-k 
        distance = a[j] - a[j-1]

        #if the distance between the points is greater than half of the fuel tank, we can't make it, so return FAIL
        if(distance*2 > c):
            return "FAIL"

        if j == n-1:  #first case for the last drop
            required = 2*distance
        else:
            required = (numTrips*2+2)*distance  #this will be the numTrips from the prior iteration.

        
        thisTrip = math.ceil(required/min(80, (c-2*distance)))    #how many trips does it take to drop the required amount of fuel at the spot, we can drop max of 80L, hence the min function?
        numTrips += thisTrip    #we need enough fuel not only to drop the fuel at the next spot, but also for any trips beyond that from drop sites further up the chain, so need numtrips to be accumulative
        
        #building array to show cost at each step of the journey.
        tripcost.append(thisTrip*2*distance + required)


    return (sum(tripcost) + c)

\end{lstlisting}
\newline

Now that we have this method, we don't have to evaluate solutions by hand anymore.
\newline\newline
Now to find the optimal solution, we again start by forcing assumption 1). Now, from the general formula for expenditure it is clear that the optimal solution minimises the amount of trips you make. Because more trips at the end of the desert compound throughout the chain of depots, leading to many more trips near the beginning, we again try to approach the problem working backwards from the end. So we try to minimise the number of trips between the last depot, and the second to last. Forcing assumptions 2) and 4), mean that we need to space the second to last depot $a_{n-1}$ one quarter of the capacity away, $\frac{c}{4}$ from the last depot $a_n$. With this spacing, we can only have to make one trip to the last depot to drop half a tank, $\frac{c}{2}$. We apply the same philosophy to the spacing between $a_{n-1}$ and $a_{n-2}$ and make this distance also $\frac{c}{4}$. We do this all the way until the very first depot, which will simply be the distance left over from repeating the process.
\newline\newline
This leads to a general solution $(a_n) = (a_0, a_1, a_2, ... a_n)$, where
\newline\begin{center}
$a_n = d - \frac{mc}{2}$,\quad$a_{k} = a_{k+1} - \frac{c}{4}$ for all $k \in [1, n-1] \subset \N$,\quad$a_0 = 0$
\end{center}
Which has been codifed in the function below:
\begin{lstlisting}
    def evenDist(d, c, m):
    last = (d - (1/2)*m*c)/m
    current = last - c/4
    a = [last]
    while(current > 0):
        a.append(current)
        current = current - c/4

    a.append(0)
    a = list(reversed(a))
    return a

\end{lstlisting}
Running this program with the input parameters in the question $(d=2413, c=140, m=12)$, we obtain the points $a = [0, 26.083, 61.083, 96.083, 131.083]$. Running the cost calculating algorithm specified earlier, we find that the expenditure of this set of points (given the standard procedure outlined earlier) is 1850.333 litres. We outline below exactly what this procedure is:
\newline
\begin{enumerate}
    \item Start with 132.2 litres, drive 26.1 litres out, drop 80 litres, come back to base empty
    \item Repeat 1) 4 more times, so there is now 400 litres stockpiled 26.1 litres into the desert
    \item Fuel up to 69.5 litres, drive 26.1 litres out, drop 17.3 litres, leaving 417.3 litres, drive back.
    \item Fuel up to full tank, drive 26.1 litres out, fuel up 26.1 litres so now have full tank again. Drive a further 35 litres to 61.1, drop 70 litres, come back, empty by the time we get to 26.1 litres, so pick up 26.1 litres. Drive back to base
    \item Repeat 4) 3 more times, leaving a total of 280 litres 61.1 litres into the desert, and 208.5 litres 26.1 litres into the desert.
    \item Fuel up, drive 26.1 out, fuel up 26.1, drive 35 more to 61.1, fuel up 35 so full tank. Drive 35 more to 96.1, drop 70 litres, come back. Empty by the time we get to 61.1, so fuel up 35, empty when we get to 26.1, so fuel up 26.1. Make it back to base empty.
    \item Repeat 6), leaving 140 litres 96.1 litres in, 140 litres 61.1 litres in, and 104.1 26.1 litres in.
    \item Fuel up, drive 26.1 out, fuel up 26.1, drive 35 more to 61.1, fuel up 35 so full tank. Drive 35 more to 96.1, fuel up 35 so full tank, drive 35 more to 131.1, drop 70. Come back, empty by the time we get to 96.1, so fuel up 35, empty by the time we get to 61.1, so fuel up 35, empty when we get to 26.1, so fuel up 26.1. Make it back to base empty. \newline
    Leaving 52 litres 26.1 litres in, 70 litres 61.1 litres in, 70 litres 96.1 litres in, and 70 litres 131.1 litres in.
    \item Fuel up, drive all the way there and back.
\end{enumerate}
The amount of fuel used is 1850.333 litres.
\newline\newline
However, through sheer trial and error with the program, we found that we could beat this with $(a) = [0,5.9575,13.9575,31.1,61.1,96.1,131.1]$, which the program tells us expends 1681.65 litres. We're not sure why this works at all, looking further into it we think it's because this solution has $(r) = [70.0, 139.99999999999997, 240.0, 239.99500000000003, 160.0, 142.98]$, which have the first two as multiples of 70, and the next 3 even multiples of 80, which gives $(t) = [1, 2, 3, 3, 2, 2]$, but we're not really sure why this is so good.




\newpage
Alternate Solutions:
\newline\newline
To give confidence in our theorised optimal solution, and also for the fun of it. We will try to beat it using some algorithmic generation of fuel depot locations.
\newline\newline
We seek a way to distribute points according to some sort of ratio between them. 
\newline\newline
Let $(a_n) = (a_0, a_1, a_2, ...a_n)$ be the locations of the fuel depots in litres from the start of the desert as in the above.
\newline\newline
Using the knowledge that the cost of dropping fuel further out increases, we decide that having the distance between fuel depots grow by some exponential function might yield a good solution. So we have $d(a_i - a_{i-1}) = g_i(e)d(a_{i-1} - a_{i-2})$. For the sake of simplicity, we assume that $g_i$ only acts on $e$ as an exponent, so we say $g_i{e} = e^{f(i)}$. We also stick with our first assumption that the last fuel depot should be as far away from the end of the desert as possible, giving $a_n = \frac{d - \frac{mc}{2}}{m}$. Putting all of these together gives: \newline
\begin{center}
    \large $a_0 = 0$,
    \newline\newline 
    \large $a_1 = \frac{d-cm}{m(1 + \sum_{k=1}^{n}e^{\sum_{j=1}^{k}f(j)})}$,
    \newline\newline \newline
    \large $a_i = a_1(1 + \sum_{k=1}^{i}e^{\sum_{j=1}^{k}f(j)}))$ for $i \in [1,n]\subset\N$
\end{center}
So now we essentially have an iterative function which distributes n fuel depots based on some exponential ratio for an arbitrary distance, capacity, and mileage. 
\newline\newline
The next question is how to choose n. Obviously we have to choose it such that the maximum distance $|a_n - a_{n-1}| \le \frac{cm}{2}$, because otherwise we would run out of fuel. From there on out it becomes a question of how much fuel is it optimal to drop on each run. For instance, it seems very wasteful to drive 65 litres out and only be able to drop 10 litres in order to make it back. From the above we suspect that this number is a fraction of our fuel tank, $\frac{c}{4}$. In practice, we will just compute the expenditure $n = 1$ through 10 and take the minimum.
\newline\newline
This distribution is automatically calculated using the functions below, with f(j) being the given exponential function.

\begin{lstlisting}
(in main...)
    for n in (range(1,10)):
        lastDrop = (d-(1/2)*m*c)/m  #half of full tank for the return journey, a full tank for the one way

        a1 = lastDrop/(eTerm(n))
        a = [0, a1]
        for i in range(2, n+2):
            a.append(a[i-1] + a1*eNum(i-1, fj))


def eTerm(n):
    result = 1
    for k in range(1,n+1):
        result += eNum(k)

    return result


def eNum(k):
    exponent = 0
    for j in range(1,k+1):
        exponent += f(j)
    
    return math.pow(math.e, exponent)   #return ln(math.e) to get rid of the e and just return the exponent (for testing out non exponential trends)

def f(j):
    #some function
    
\end{lstlisting}
Now, we calculate expenditure using the same code as before, and we observe the data for many different functions f(j). This data is plotted below, with distance on the x axis and expenditure on the y.
\newline\newline
\includegraphics[scale=0.1]{images/graph1.jpg}
\includegraphics[scale=0.1]{images/graph2.jpg}
\newline\newline
We can see that the solution which we proposed as our optimal general solution (grey) is indeed optimal most of the time. However, there are some solutions which appear better for certain distances. For example, the solution distributed with a factor of $e^{\frac{1}{n^2}}$ looks to perform substantially better around $d = 3800km$.
\end{document}

